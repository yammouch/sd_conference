\documentclass[twocolumn]{article}
%\usepackage{sasimi}
%%%% If you use A4 size paper add ``a4'' option as follows
\usepackage[a4]{sasimi}

\usepackage{color}
%%%% Optional package for better font
%\usepackage{txfonts}

\begin{document}
%date not printed
\date{}

%make title
\title{\Large\bf
Author's GUIDE\\~\\
\large\bf				       
Preparation of Papers in Two-Column Format\\
for \sasimititle}
							    
%%%%%%%%%%%%%%%%%%%%%%%%%%%%%%%%%%%%%%%%%%%%%%%%%%%%%%%%%%%%%%%%%%%%%%%
%%%%                                                               %%%%
%%%% ATTENTION: To permit blind review,                            %%%%
%%%%            Please DO NOT include author names                 %%%%
%%%%            in initial submission                              %%%%
%%%%                                                               %%%%
%%%%%%%%%%%%%%%%%%%%%%%%%%%%%%%%%%%%%%%%%%%%%%%%%%%%%%%%%%%%%%%%%%%%%%%

%for single author
%\author{Center the Authors Names Here \\
%Center the Affiliations Here\\
%Center the City, Stats and Country Here\\     
%{\small (it is your option if you want your entire address listed)}}

%for two authors
\author{\normalsize
 \begin{tabular}[t]{c@{\extracolsep{8em}}c}
%  \large Author Name& \large Coauthor Name \\
%  \\
%   Author Department & Coauthor Department \\
%   Author Institute  & Coauthor Institute \\
%   City, ST~~zipcode & City, ST~~zipcode \\
%   author@dept.inst.or.jp & coa@dept.inst.ac.jp
\\
\\
\\
\\
\\
\end{tabular}}

\maketitle

\thispagestyle{empty}

{\small\bf Abstract---
Abstract is a brief (50-80 word) synopsis of your paper. 
The purpose is to provide a quick outline of your presentation, 
giving the reader an overview of the research. 
It must be fit within the size allowed, which is about 3 inches or 7.5 centimeters.}

\section{Introduction}
						    
These introductions give you basic guidelines for
preparing initial-submission papers for the SASIMI 2018.
The instructions assume that you have computer desktop publishing equipment with several fonts. 

These instructions have been prepared in the preferred format. 
For items not addressed here, please refer to recent issues of IEEE Transactions and simulate,
as closely as possible.

\section{How to Format the Page}

\subsection{Full-Size Camera-Ready Copy}

Prepare Camera-Ready paper in full size format, on A4 size or 8 1/2''x 11'' (21.5 cm x 27.9 cm) paper.

The length of a short paper is 2 pages, and that of a full paper is 3 to 6 pages, including all figures and tables.


\subsection{Fonts}

The best results will be obtained if your computer word-processor has several font sizes. 
Try to follow the font sizes specified in Table I as best as you can. 
As an aid to gauging font size, 1 point is about 0.35 mm. 
Use a proportional, serif font such as Times of Dutch Roman. 
Any paper using a font smaller than 9~pt or larger than 10~pt for main text will not be included in the proceedings.

\begin{table}[tb]
\caption{Fonts for Camera-Ready Papers}		     
\begin{minipage}{8cm}
\def\arraystretch{1.5}\tabcolsep 2pt
\def\thefootnote{a}\footnotesize
\begin{tabular}{l@{~}l@{~~~}l}
\hline
\parbox[c]{7mm}{Font\newline Size} & Style & Text\\
\hline
 14pt&bold     &Paper title\\
 12pt&         &Authors' names\\
 10pt&         &Authors' affiliations, main text, equations,\\[-5pt]
     &         &first letters in section titles\footnotemark[1]\\
 10pt&italic   &Subheadings\\
 ~9pt&bold     &Abstract\\
 ~8pt&         &Section titles\footnotemark[1], table
                names\footnotemark[1], first letters in table\\[-5pt]
     &         &captions\footnotemark[1],
                tables, figure captions, references,\\[-5pt]
     &         &footnotes, text subscripts and superscripts\\
 ~6pt&         &Table captions\footnotemark[1], table superscripts\\
\hline
\end{tabular}
\footnotetext[1]{\scriptsize Uppercase}
\end{minipage}
\end{table}				   


\subsection{Formats}

In formatting your A4-size paper, set top margin to 29 mm (1.14 inches), 
bottom margin to 30 mm (1.18 inches), left and right margins to 15 mm (0.59 inches).
If you are using paper 8 1/2'' x 11'', set the top margin to 17 mm (0.67 inches), 
bottom margin to 24 mm (0.94 inches), the left to 18mm (0.71 inches) and right margins to 17 mm (0.67 inches). 
The column width is 88 mm (3.46 inches) with 5 mm (0.2 inches) space between the two columns. 
You should left- and right-justify your columns. 

All of the accepted papers will be included in electronic proceedings. 
In addition, SASIMI 2018 is going to introduce an ``official archive''
on the web site (\verb|http://sasimi.jp/|) where only the papers chosen by the authors are uploaded for open access. 
Please keep the above margins since papers are standardized on US letter-size paper in both cases.
Use automatic hyphenation if you have it and check spelling. Either digitize or paste down your figures.

\iffalse
\subsection{Reduction Mats}

If you have only typewriter fonts available, use 75\% reduction mats
(model paper). The mats are not available in this conference.  If
necessary, make your own mats, 27.9cm X 36.5 cm. If you define the lower
left corner on the mat as the origin, the (x,y) coordinate in centimeters
of the column corners are given in Table II. Attach paragraphs and
figures with paste. It is not necessary to right-justify your columns.
If you do not have italics, use underlines. Do not use a dot-matrix
printer. Avoid hand lettering. If you prepare the mat text with a
computer, rather than a typewriter, final type size after reduction
should approximate those listed in Table I.

%Table I has been described using macros (re-)defined in this file,
%but you can describe tables by a standard ``center evironment of LaTeX''
%as shown below.

\begin{center}
{\footnotesize\sc TABLE II\\
 Coordinates in Centimeters of Column\\
 Corners for Reduction Mats}
\vskip .9\baselineskip
{\def\arraystretch{1.5}\tabcolsep 5pt\footnotesize
\begin{tabular}{ll}
\hline
First Column   & Second Column\\
\hline
(1.95, 34.50)(13.65, 34.50)&(14.35, 34.50)(25.05, 34.50)\\
(1.95,  1.80)(13.65,  1.80)&(14.35,  1.80)(26.05,  1.80)\\
\hline
\end{tabular}}
\end{center}
\fi
							    
\section{Figures and Tables}

Place figures and tables at the tops and bottoms of columns. 
Avoid placing them in the middle of columns. 
Large figures and tables may span across both columns. 
Color figures can be included in proceedings, 
which should be visible even when the manuscript is printed in monochrome.
Figure captions should be below the figures; table captions should be above the tables. 
Avoid placing figures and tables before their first mention in the text. 
Use the abbreviation ``Fig.1'', even at the beginning of a sentence.

\begin{figure}[tb]
\begin{center}
\begin{minipage}{5cm}
$.$\hrulefill $.$\\$|$\hfill $|$\\$|$\hfill $|$\\$|$\hfill $|$\\
$|$\hfill this is \hfill $|$\\
$|$\hfill a sample \hfill $|$\\
$|$\hfill  figure  \hfill $|$\\
$|$\hfill $|$\\$|$\hfill $|$\\$|$\hfill $|$\\$.$\hrulefill $.$\\
\end{minipage}
\caption{This is a sample figure. Captions exceeding
one line are arranged like this.}
\end{center}
\end{figure}

\section{Helpful Hints}

\subsection{References}
List and number all references at the end of the paper. When referring
to them in the text, type the corresponding reference number in the
parentheses as shown at the end of this sentence \cite{key}. Number
the citations consecutively. The sentence punctuation follows the
parentheses. Do not use ``Ref.\cite{baz}'' or
``reference\cite{baz}'' except at the beginning of a sentence.

\subsection{Footnotes}
Number the footnotes separately in superscripts. Place the actual
footnote at the bottom of the column in which it is cited. Do not put
footnotes in the reference list.

\subsection{Authors names}

Give all authors' names; do not use ``et al'' unless there are six
authors or more. Papers that have not been published, even if they have
been submitted for publication, should be cited as
``unpublished''\cite{unpub}.  Papers that have been accepted for
publication should be cited as ``in press''\cite{inpress}.
Capitalize only the first word in a paper title, except for proper nouns
and element symbols.

For papers published in translation journals, please give the English
citation first, followed by the original foreign language
citations\cite{trans}.


\section{Summary and Conclusions}
This template will get you through a sample article.  The template
will be available on the web-site \verb|http://sasimi.jp/|.


%%%%%%%%%%%%%%%%%%%%%%%%%%%%%%%%%%%%%%%%%%%%%%%%%%%%%%%%%%%%%%%%%%%%%%%
%%%%                                                               %%%%
%%%% ATTENTION: To permit blind review,                            %%%%
%%%%            Please DO NOT include acknowledgement              %%%%
%%%%            in initial submission                              %%%%
%%%%                                                               %%%%
%%%%%%%%%%%%%%%%%%%%%%%%%%%%%%%%%%%%%%%%%%%%%%%%%%%%%%%%%%%%%%%%%%%%%%%

%this is how to do an unnumbered subsection
%\section*{\sc Acknowledgements}
%This guide is compiled from the author's Guide for the
%ASP-DAC'95/CHDL'95/VLSI'95. 
%This article was written by referring to {\em ``Author's guide --
%Preparation of Papers in Two-Column Format for VLSI Symposia on
%Technology and Circuits''}, the {\em ``Preparation of Papers in
%Two-Column Format for the Proceedings of the 32nd ACM/IEEE Design
%Automation Conference''} written by Ann Burgmeyer, IEEE and {\em ``the
%template for producing IEEE-format articles using LaTeX''}, written by
%Matthew Ward, Worcester Polytechnic Institute.


\begin{thebibliography}{9}
\footnotesize
\bibitem{key}
I. M. Author,
``Some related article I wrote,''
{\em Some Fine Journal}, Vol. 17, pp. 1-100, 1987.

\bibitem{baz}
A. N. Expert,
{\em A Book He Wrote,}				     
His Publisher, 1989.

\bibitem{unpub}
M. Smith,
``Title of paper optional here,''
unpublished.

\bibitem{inpress}
K. Rose,
``Title of paper with only first word capitalized,''
in press.

\bibitem{trans}
T. Murayama,
``Title of paper published in translation journals,''
{\em Some English Journal}, Vol. 17, pp. 1-100, 1995.
({\em Original Foreign Journal, Vol. 1, pp. 100-200, 1993.})


\end{thebibliography}
\end{document}
